\chapter*{}
%\thispagestyle{empty}
%\cleardoublepage

%\thispagestyle{empty}

\input{portada/portada_2}



\cleardoublepage
\thispagestyle{empty}

\begin{center}
{\large\bfseries Controlador Neuronal: Esquemas de aprendizaje de modelos internos de brazo robótico con múltiples articulaciones}\\
\end{center}
\begin{center}
Julio Jesús Vizcaíno Molina\\
\end{center}

%\vspace{0.7cm}
\noindent{\textbf{Palabras clave}: robot\_controller, neural\_networks, deep\_learning, baxter\_robot}\\

\vspace{0.7cm}
\noindent{\textbf{Resumen}}\\

Calcular el modelo dinámico de un brazo robótico es la mejor manera de entender y diseñar un controlador para dicho brazo. Sin embargo, es una tarea complicada que se basa en el conocimiento del funcionamiento del robot, así como de sus características. El controlador que aquí se presenta resuelve estos problemas. Se basa en el auto-aprendizaje del modelo, así como de los parámetros del robot. Para ello hace uso de técnicas de aprendizaje automático (Deep Learning). Estas redes son sistemas de propósito general que parametrizan variables internas en la fase de entrenamiento, para así obtener un modelo concreto al final de esta fase. Este modelo es capaz de desempeñar el papel del controlador empleado para modelarlo sobre datos no visto antes.
\cleardoublepage


\thispagestyle{empty}


\begin{center}
{\large\bfseries Neural Controller: Project Subtitle}\\
\end{center}
\begin{center}
Julio Jesús Vizcaíno Molina\\
\end{center}

%\vspace{0.7cm}
\noindent{\textbf{Keywords}: Keyword1, Keyword2, Keyword3, ....}\\

\vspace{0.7cm}
\noindent{\textbf{Abstract}}\\

Write here the abstract in English.

\chapter*{}
\thispagestyle{empty}

\noindent\rule[-1ex]{\textwidth}{2pt}\\[4.5ex]

Yo, \textbf{Julio Jesús Vizcaíno Molina}, alumno de la titulación Grado en Ingeniería de Tecnologías de Telecomunicación de la \textbf{Escuela Técnica Superior de Ingenierías Informática y de Telecomunicación de la Universidad de Granada}, con DNI 77151856n, autorizo la ubicación de la siguiente copia de mi Trabajo Fin de Grado en la biblioteca del centro para que pueda ser consultada por las personas que lo deseen.

\vspace{6cm}

\noindent Fdo: Julio Jesús Vizcaíno Molina

\vspace{2cm}

\begin{flushright}
Granada a hoy de este mes de 2016
\end{flushright}


\chapter*{}
\thispagestyle{empty}

\noindent\rule[-1ex]{\textwidth}{2pt}\\[4.5ex]

D. \textbf{Jesús Garrido Alcázar}, Profesor del Área de XXXX del Departamento ATC de la Universidad de Granada.

\vspace{0.5cm}

D.ª \textbf{Eva Martínez Ortigosa}, Profesora del Área de XXXX del Departamento ATC de la Universidad de Granada.


\vspace{0.5cm}

\textbf{Informan:}

\vspace{0.5cm}

Que el presente trabajo, titulado \textit{\textbf{Controlador Neuronal, Esquemas de aprendizaje de modelos internos de brazo robótico con múltiples articulaciones}}, ha sido realizado bajo su supervisión por \textbf{Julio Jesús Vizcaíno Molina}, y autorizamos la defensa de dicho trabajo ante el tribunal que corresponda.

\vspace{0.5cm}

Y para que conste, expiden y firman el presente informe en Granada a X de mes de 201 .

\vspace{1cm}

\textbf{Los directores:}

\vspace{5cm}

\noindent \textbf{Jesús Garrido Alcázar \ \ \ \ \ Eva Martínez Ortigosa}

\chapter*{Agradecimientos}
\thispagestyle{empty}

       \vspace{1cm}


Poner aquí agradecimientos...

