\chapter*{}
%\thispagestyle{empty}
%\cleardoublepage

%\thispagestyle{empty}

\input{portada/portada_2}



\cleardoublepage
\thispagestyle{empty}

\begin{center}
{\large\bfseries Controlador Neuronal: Esquemas de aprendizaje de modelos internos de brazo robótico con múltiples articulaciones}\\
\end{center}
\begin{center}
Julio Jesús Vizcaíno Molina\\
\end{center}

%\vspace{0.7cm}
\noindent{\textbf{Palabras clave}: control\_robótico, redes\_neuronales, aprendizaje\_profundo, robot\_baxter, ros, modelo\_interno}\\

\vspace{0.7cm}
\noindent{\textbf{Resumen}}\\

Una nueva generación de robots de baja fuerza está implantándose en la sociedad. Su primer paso lo dan en las industrias donde se requiere la interacción con personas. Este tipo de robots requiere de un control distinto de los usados en los típicos robots industriales de gran fuerza. Para ello, es necesario un conocimiento del modelo interno del robot, de las fuerzas que aplican y los movimientos que éstas generan.

El modelo interno puede analizarse y ajustar los parámetros del mismo con la experimentación, pero es un proceso complejo y dependiente tanto del robot concreto como del estado del mismo. Cualquier cambio realizado en el mismo requerirá un rediseño del mismo.

La opción propuesta en este trabajo consiste en aprender el modelo automáticamente, mediante el uso de técnicas de aprendizaje automático. En concreto, se usarán las técnicas del aprendizaje profundo, el cual consiste en modelos muy complejos que aprenden de los datos que se le entregan, ofreciendo resultados en algunos campos nunca antes obtenidos por otro tipo de algoritmos de aprendizaje automático.

Por otro lado, el desarrollo robótico ha visto en la historia una gran limitación, y es que las implementaciones realizadas para un robot no pueden utilizarse para otros. El sistema operativo para robots \texttt{ROS} ofrece una solución a dicho problema, ya que pone a disposición del desarrollador las herramientas necesarias para crear elementos individuales que después se podrán conectar entre sí.

\cleardoublepage


\thispagestyle{empty}


\begin{center}
{\large\bfseries Neural Controller: Learning schemes of internal robotic arm models with multiple joints}\\
\end{center}
\begin{center}
Julio Jesús Vizcaíno Molina\\
\end{center}

%\vspace{0.7cm}
\noindent{\textbf{Palabras clave}: robot\_control, neural\_networks, deep\_learning, baxter\_robot, ros, internal\_model}\\

\vspace{0.7cm}
\noindent{\textbf{Abstract}}\\

A new low force robot generation is arriving. First, they are being implemented in industries where human relation with robots is necessary. This kind of robots require different control systems in order to achieve simylar results than achieved by tipycal high force robots. Thus, it is needed a known internal model of the robot, this is, forces applied by the robot and positions reached with them.

The desired internal model can be extracted by analysing and tuning its parametres with testing, but it is usually a hard and dependent of the robot way. Any change made in the robot will affect to the internal model, and must be redesinged.

This work proposal consists in learning the model automatically, using machine learning techniques. Specifically, it uses deep learning, which state of the art techniques of machine learning. They are very long models capable of learning difficult models only viewing data.

On the other hand, history has shown that robot developing is a low and limited scenary, due to repetition of very hard implementations in each robot created. Each team had to develop all functionalities of the robot from the ground, which is exhausting and doesn't let evolution becomes. Robot Operative System (\texttt{ROS}) is a operative system design to robot development which solves such problems by developing small pieces of code which can be connected with each other.

\chapter*{}
\thispagestyle{empty}

\noindent\rule[-1ex]{\textwidth}{2pt}\\[4.5ex]

Yo, \textbf{Julio Jesús Vizcaíno Molina}, alumno de la titulación Grado en Ingeniería de Tecnologías de Telecomunicación de la \textbf{Escuela Técnica Superior de Ingenierías Informática y de Telecomunicación de la Universidad de Granada}, con DNI 77151856n, autorizo la ubicación de la siguiente copia de mi Trabajo Fin de Grado en la biblioteca del centro para que pueda ser consultada por las personas que lo deseen.

\vspace{6cm}

\noindent Fdo: Julio Jesús Vizcaíno Molina

\vspace{2cm}

\begin{flushright}
Granada a 12 de Diciembre de 2016
\end{flushright}


\chapter*{}
\thispagestyle{empty}

\noindent\rule[-1ex]{\textwidth}{2pt}\\[4.5ex]

D. \textbf{Jesús Garrido Alcázar}, Profesor del Área de Arquitectura y Tecnología de Computadores del Departamento Arquitectura y Tecnología de Computadores de la Universidad de Granada.

\vspace{0.5cm}

D.ª \textbf{Eva Martínez Ortigosa}, Profesora del Área de Arquitectura y Tecnología de Computadores del Departamento de Arquitectura y Tecnología de Computadores de la Universidad de Granada.


\vspace{0.5cm}

\textbf{Informan:}

\vspace{0.5cm}

Que el presente trabajo, titulado \textit{\textbf{Controlador Neuronal, Esquemas de aprendizaje de modelos internos de brazo robótico con múltiples articulaciones}}, ha sido realizado bajo su supervisión por \textbf{Julio Jesús Vizcaíno Molina}, y autorizamos la defensa de dicho trabajo ante el tribunal que corresponda.

\vspace{0.5cm}

Y para que conste, expiden y firman el presente informe en Granada a 12 de Diciembre de 2016.

\vspace{1cm}

\textbf{Los directores:}

\vspace{5cm}

\noindent \textbf{Jesús Garrido Alcázar \ \ \ \ \ Eva Martínez Ortigosa}

\chapter*{Agradecimientos}
\thispagestyle{empty}

       \vspace{1cm}

A mis tutores Jesús, Eva y Eduardo por su paciencia y directrices; a mis padres, que todavía no tienen claro a qué me dedico; a mi hermana y su temor a que los robots controlen el mundo; y a mis amigos, cuyas bromas sobre mi trabajo hacen reír en el momento, y ayudan a seguir adelante cuando las cosas se tuercen.

