\chapter{Conclusiones}
A continuación se exponen las conclusiones extraídas del proyecto.
\section{Trabajo realizado}
En este trabajo fin de grado se han estudiado nuevos paradigmas de control generados por la necesidad de los mismos, debido a la implantación de robots de baja fuerza en el mercado.

Se han estudiado y utilizado las herramientas necesarias para desarrollar un producto en el marco de la robótica, y se ha diseñado un controlador anticipativo haciendo uso de técnicas de estado del arte en aprendizaje profundo.

En primer lugar, se estudió el lenguaje de programación Python, que ha acompañado durante todo el proyecto.

En paralelo, se aprendió a usar y entender el sistema operativo \ros, que permite desarrollar aplicaciones robóticas de manera individualizada, para luego juntarlas entre sí. Además, ofrece herramientas externas a la programación de módulos que permiten la interacción con los mismos, además de ofrecer una solución a la distribución y compatibilidad de paquetes.

Se estudió el simulador V-REP, así como el simulador Gazebo, y se entendieron sus características y limitaciones.

Se ha enfrentado un problema real, el de obtener un controlador que aprenda las dinámicas internas de un robot basado en aprendizaje automático. Este problema real ha conducido el trabajo por distintos senderos hasta llegar a la solución finalmente alcanzada.

Se han estudiado los modos de control que ofrece el robot biomórfico Baxter, así como todas las características que éste ofrece.

Se ha estudiado la integración de este robot con \ros, y se ha descubierto en la API de Python de Baxter una manera sencilla de interactuar con el mismo.

Se han estudiado los distintos sistemas de control, y se ha implementado un controlador PID. Se han obtenido los parámetros que optimizan ese PID y se han encontrado las limitaciones del mismo.

Se han estudiado técnicas de optimización de parámetros para adquirir los parámetros que mejoran la respuesta del PID, y se ha encontrado en ellos la limitación temporal que supone utilizarlos en una plataforma real, no simulada.

Se ha extraído y procesado una base de datos real de movimientos del Baxter.

Se han estudiado las técnicas usadas en la actualidad en el aprendizaje automático, en concreto, aprendizaje profundo. Se han estudiado las virtudes y limitaciones de dichos sistemas y se ha realizado un experimento con datos reales.

Se han obtenido resultados de dicha red y se han analizado los problemas encontrados en dichos resultados.

Finalmente, se ha afrontado un proyecto autónomo y se han enfrentado las decisiones de diseño y la organización del mismo. Por suerte, se ha contado con la ayuda de los tutores.
\section{Objetivos alcanzados}
Al comienzo del trabajo se propusieron los siguientes objetivos:


\section{Trabajo futuro}
% Crear un módulo en ROS que haga uso del controlador