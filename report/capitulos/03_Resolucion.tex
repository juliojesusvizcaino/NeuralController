% Simuladores utilizados
% Por qué se dejó de usar el simulador
%\begin{enumerate}
%	\item Estudio de los mecanismos de control implementados en el robot biomórfico Baxter.
%	\item Caracterización y obtención de una base de datos de movimientos (trayectorias y torques aplicados) utilizando los controladores incluidos en el robot.
%	\item Estudio de la viabilidad de implementación de un sistema de control adaptativo basado en técnicas de machine learning (pos ejemplo, redes neuronales hacia-adelante o recurrentes).
%\end{enumerate}
%Finalmente se abordará una etapa de extracción de resultados y evaluación del modelo desarrollado en el marco de tareas de movimientos de alcance de un objetivo.

\chapter{Resolución}
\section{Desarrollo}
\subsection{Mecanismos de control en Baxter}


	

\subsection{Obtención de datos}



\subsubsection{rosbag}

\section{Resultados}

% explicar pid
% explicar controlador feedforward
% explicar error distal
% contar historia del proyecto:
% 	- obtener parámetros óptimos del PID

% ros: mensajes con marcas temporales y secuencias
% rosbag

% disposición del robot, conectado directamente al ordenador (y no a un router)
% Referenciar imágenes pertenecientes a la página de baxter
% ¿Números con números, o con letras?
% Cómo escribir Hz correctamente (estilo latex)
% Cómo referirme al ordenador con el que controlo el Baxter (lo estoy llamando ordenador)
% Cómo referirme al Baxter (lo llamo Baxter, robot, robot Baxter)
% Utilizar la voz pasiva (se hizo, se ejecutó...), o la primera persona del plural (hicimos, ejecutamos...)?