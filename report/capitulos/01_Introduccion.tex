\chapter{Introducción}
\section{Motivación}
El futuro tecnológico sugiere que, al igual que hoy día es normal tener un teléfono que nos mantiene conectados, los robots serán una parte fundamental de la realidad que nos tocará vivir. Esta interacción entre humano y máquina, requiere de ciertas medidas de control, a fin de que no sean peligrosos para nosotros.

Se podrían diseñar métodos que basados en la situación, se comportaran siempre de manera segura. Pero, como cualquier método que se base en la información, falla cuando esta información no sigue los patrones establecidos o no se tienen en cuenta toda la información posible. Es por esto por lo que se requiere que la fuerza de estos robots está limitada.

Esto ha conducido a la construcción de robots de baja fuerza por parte de los fabricantes. Un ejemplo de ello es el robot biomórfico Baxter \cite{baxter}. Este robot está pensado para trabajar en industrias donde comparte el espacio con personas, o interactúa con ellas.

El control de este tipo de robots \cite{kulakowski2007dynamic}\cite{brosilow2002techniques}\cite{tolu2013adaptive}\cite{tolu2012bio}  también ha de cambiar, debido a que, por la fuerza limitada que ofrecen, cualquier fuerza interna o externa modificaría el comportamiento del robot. Es por esto por lo que se requiere un mayor conocimiento del modelo dinámico de dichos robots.

Este conocimiento se puede adquirir del análisis del robot, pero es un proceso complicado que depende de cada robot y de la situación en la que se encuentre el mismo. Es por esto por lo que, un nuevo enfoque basado en el aprendizaje automático \cite{andrewng}\cite{abu2012learning} puede dar un valor añadido al diseño de este tipo de sistemas.

En concreto, el aprendizaje profundo \cite{udacitydeeplearning}\cite{hinton} comprende una serie de técnicas que supone el estado del arte de la inteligencia artificial. Está obteniendo resultados nunca antes conseguidos por ningún otro algoritmo en, por ejemplo, los campos de reconocimiento de objetos, reconocimiento automático del habla, e incluso entendimiento del lenguaje natural.

Para llegar a ese futuro en el que conviviremos con los robots, no es suficiente con que un sólo equipo realice todo el trabajo para un solo robot. Este es el enfoque que se ha aplicado durante años, y con el cual no se han conseguido realizar mas que tareas sencillas. Un nuevo enfoque es necesario. Uno descentralizado, donde se puedan crear módulos compatibles entre sí y que no se requiera desarrollar tantas veces como proyectos se hagan. Éste es el enfoque que sigue el sistema operativo para robots \texttt{ROS} (Robot Operative System) \cite{ros}.

\section{Objetivos}
Los objetivos del trabajo son los siguientes:
\begin{enumerate}
	\item Estudio de los mecanismos de control implementados en el robot biomórfico Baxter.
	\item Caracterización y obtención de una base de datos de movimientos utilizando los controladores incluidos en el robot.
	\item Estudio de la viabilidad de implementación de un sistema de control adaptativo basado en técnicas de machine learning.
\end{enumerate}
\section{Metodología}
Para alcanzar estos objetivos, se realizará una división temporal del problema.

\begin{enumerate}
	\item  Primero, se realizará una etapa de aprendizaje y familiarización de los elementos de los que consta el proyecto.
	
	\item  A continuación, le seguirá una etapa de experimentación y extracción de resultados. Esta etapa y la anterior se mezclaran conforme se vayan descubriendo los requerimientos no planificados de los experimentos.
	
	\item  Para terminar, se analizarán los resultados y se sacarán las conclusiones oportunas.
\end{enumerate}
\section{Organización de la memoria}
De igual manera que la descrita en la metodología, la memoria se organiza en tres bloques.

\subsection{Materiales y métodos}
En este bloque se describirán todos los materiales y métodos usados en la práctica. Se hablará de \texttt{ROS} y de Baxter. También se hablará del aprendizaje automático y de los modelos de control robótico.

\subsection{Diseño experimental y resultados}
En este bloque se explicará la experimentación realizada, así como las decisiones tomadas acerca de la misma. Al final de la misma se mostrarán los resultados obtenidos en los experimentos.

\subsection{Conclusiones}
Se terminará la memoria con las conclusiones, el trabajo realizado y las líneas de continuación del presente trabajo.