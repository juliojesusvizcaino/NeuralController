\section{Red Neuronal}
\subsection{Definición}
Una red neuronal artificial es un conjunto de algoritmos de aprendizaje automático capaces de extraer modelos a partir de un conjunto de datos de aprendizaje. De esta manera, estos sistemas son capaces de emular la fuente generadora de datos y producir salidas coherentes a partir de entradas no vistas con anterioridad.
\subsection{Arquitecturas}
En función de la topología de la red, se contemplan dos tipos fundamentales:
\subsubsection{Propagación hacia delante}
Las conexiones entre las neuronas es de un solo sentido, de modo que no se forman bucles entre ninguna de las neuronas (fig. \ref{imagenes/apendices/red_neuronal/feed_forward})