\section{Red Neuronal}
\subsection{Definición}
Una red neuronal artificial es un conjunto de algoritmos de aprendizaje automático capaces de extraer modelos a partir de un conjunto de datos de aprendizaje. De esta manera, estos sistemas son capaces de emular la fuente generadora de datos y producir salidas coherentes a partir de entradas no vistas con anterioridad.
\subsection{Arquitecturas}
En función de la topología de la red, se contemplan dos tipos fundamentales:
\subsubsection{Propagación hacia delante}
Las conexiones entre las neuronas es de un solo sentido, de modo que no se forman bucles entre ninguna de las neuronas (figura \ref{fig:apendices/feedforward})

\begin{figure}
	\centering
	\begin{neuralnetwork}[height=4.7]
		\newcommand{\nodetextclear}[2]{}
		\newcommand{\nodetextx}[2]{$x_#2$}
		\newcommand{\nodetexty}[2]{$y_#2$}
		\inputlayer[count=4, bias=false, title=Capa de\\entrada, text=\nodetextx]
		\hiddenlayer[count=5, bias=false, title=Capa\\oculta, text=\nodetextclear] \linklayers
		\outputlayer[count=3, title=Capa de\\salida, text=\nodetexty] \linklayers
	\end{neuralnetwork}
	\caption{Red neuronal de propagación hacia adelante}
	\label{fig:apendices/feedforward}
\end{figure}

